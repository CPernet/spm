\chapter{3D source reconstruction: Imaging approach \label{Chap:eeg:imaging}}

Here is a brief help to the 3D reconstruction based on the Imaging approach.

\section{Introduction}
\label{sec:imaginv_intro}
This chapter focuses on the imaging (or distributed) method for doing 
EEG/MEG source reconstruction in SPM.
Such an approach to spatial projection onto (3D) brain space consists
in considering a large amount of dipolar sources all over the cortical sheet, 
with fixed locations and orientations. This renders the observation model linear, 
the unknown variables being the source amplitudes or power.\\
Given epoched and preprocessed data (see chapter \ref{Chap:eeg:preprocessing}), the evoked and/or induced
activity for each dipolar source can be estimated, for a single time-sample or a 
wider peristimulus time window.\\
The obtained reconstructed activity is in 3D voxel space and can be further analyzed using mass-univariate analysis in SPM.

Contrary to PET/fMRI data reconstruction, EEG/MEG source reconstruction 
is a non trivial operation. Often compared to estimating a body shape from 
its shadow, inferring brain activity from scalp data is mathematically ill-posed 
and requires prior information such as anatomical, functional or mathematical
constraints to isolate a unique and most probable solution~\cite{Baillet01}.

Distributed linear models have been around for more than a decade now~\cite{Dale93}
and the proposed pipeline in SPM for imaging solution is classical and very similar
to common approaches in the field. However, at least two aspects are quite original 
and should be emphasized here:

\begin{itemize}
\item Based on an empirical Bayesian formalism, the inversion is meant 
to be generic in the sense it can incorporate and estimate the relevance
of multiple constraints of various nature; data-driven relevance estimation 
being made possible through Bayesian model 
comparison~\cite{peb1,cp_empirical_eeg,jm_multiple,karl_induced}.
\item The subject's specific anatomy is incorporated in the generative model
of the data, in a fashion that eschews individual cortical surface extraction.
The individual cortical mesh is obtained automatically from a canonical mesh in 
MNI space, providing a simple and efficient way of reporting results in stereotactic coordinates.
\end{itemize}

The EEG/MEG imaging pipeline is divided into four consecutive steps which characterize
any inverse procedure with an additional step of summarizing the results. In this chapter, we go through each of those steps that all need 
to be completed when proceeding with a full inverse analysis:

\begin{enumerate}
    \item Source space modeling,
    \item Data co-registration,
    \item Forward computation,
    \item Inverse reconstruction.
    \item Summarizing the results of inverse reconstruction as an image.
\end{enumerate}

Whereas the three first steps are part of the whole generative model, inverse reconstruction
step consists in the Bayesian inversion and is the only one involving the actual EEG/MEG data.\\

\section{Getting started}

Everything which is described hereafter is accessible from SPM user-interface by choosing 
the 'EEG' application, '3D Source Reconstr.' button. When you press this button a new window
will appear with a GUI that will guide you through the necessary steps to obtain an imaging
reconstruction of your data. At each step, the buttons that are not yet relevant for this step
will be disabled. When you open the window the only two buttons you can press are 'Load' which enables you
to load a pre-processed SPM M\\EEG dataset and the 'Group inversion' button that will be described below. 
You can load into 3D source reconstruction a dataset which is either epoched with single trials for different
conditions, averaged with one event related potential (ERP) per condition or grand-averaged. An important pre-condition
for loading a dataset is that it should contain sensors and fiducials. This will be checked when you load a file
and loading will fail in case of a problem. You should make sure that the file you are loading
contains channels of only one modality (either EEG or MEG) and for this modality there is a sensor description. 
Datasets with both EEG and MEG sensors will be supported in the future, but not yet in SPM8 beta.
If, for instance, you have a MEG dataset with some EEG channels, change their type to 'Other' before trying to load it.
MEG datasets converted by SPM from their raw formats will always contain
sensor and fiducial descriptions. In case of EEG for some supported channel setups (such as extended 1020 or Biosemi)
SPM will provide default channel locations and fiducials that you might or might not want to use for your reconstruction.
Sensor and fiducial descriptions can be modified using the 'Prepare' interface (\texttt{spm\_eeg\_prep}) and in this
interface you can also verify that these descriptions are sensible by performing a coregistration (see chapter \ref{Chap:eeg:preprocessing} and also below for more details about coregistration). 

When you successfully load a dataset you are asked to give a name to the present analysis cell. In SPM it is possible to
perform multiple reconstructions of the same dataset with different parameters. The results of these reconstructions
will be stored with the dataset if you press the 'Save' button. They can be loaded an reviewed again using
the 3D GUI and also with the SPM EEG reviewing tool. From the command line you can access source reconstruction
results via the \texttt{D.inv} field of the meeg object. This field (if present) is a cell array of structures and does not require
methods to access and modify it. Each cell contains the results of a different reconstruction. In the GUI you
can navigate between these cells using the buttons in the second row. You can also create, delete and clear cells.
The label you input at the beginning will be attached to the cell for you to identify it.


\section{Source space modeling}

After entering the label you will see the 'Template' and 'MRI' button enabled. The 'MRI' button will 
create an individual head model based on the subject's structural scan. This is presently only supported for MEG
data. SPM will ask for the subject's structural image. It might take some time to prepare the model as the image
should be segmented. Individual meshes for the inner-skull and scalp surfaces are then computed from the
segmented image. They are obtained by performing a binary mask of the the volumes 
delimited by the inner-skull and scalp surface respectively. Then, using an initial 
spherical mesh, a realistic-shaped mesh is obtained for each of the two tissues 
and further regularized via an erosion and growing procedure. The code behind this button is work 
in progress and it is advised to examine the resulting
model to see that the boundaries of the scalp and the inner-skull compartment look sensible. 

Presently we advise to use the 'Template' button for both EEG and MEG. This button will use SPM's template head
model based on MNI brain. The corresponding structural image can be found under \texttt{EEGtemplates\\smri.nii} in SPM
directory. When you use the template different things will happen depending on whether your data is EEG or MEG.
For EEG your electrode positions will be transformed to match the template head. So even if your subject's head
is quite different from the template, you should be able to get good results. For MEG the template head will
be transformed to match the fiducials and headshape that come with the MEG data. In this case having a headshape
measurement can be quite helpful to provide SPM with more data to scale the head correclty. From the user
perspective the two options will look quite similar.

No matter whether 'MRI' or 'Template' button was used the cortical mesh, which describes the locations of
possible sources of EEG and MEG signal is obtained from a template mesh.
In case of EEG the mesh is used as is, and in case of MEG it is transformed with the head model. 
Four mesh sizes are available (3004, 4004, 5004 and 7204 vertices) and you need to choose the size
you want to work with. It is advised to work with the highest resolution mesh if possible. The inversion
process might be slightly slower with higher resolution mesh, but the solution quality is much better.

\section{Coregistration}

In order for SPM to provide a meaningful
interpretation of the results of source reconstruction, it should link the coordinate
system in which sensor positions are originally represented to the coordinate system of
a structural MRI image (MNI coordinates). In general to link between two coordinate
systems you will need a set of at least 3 points whose coordinates are known in both systems.
This is a kind of 'Rosetta stone' that can be used to convert a position of any point from one system
to the other. These points are called 'fiducials' and the process of providing SPM with all the necessary
information to create the 'Rosetta stone' for your data is called 'coregistration'.  \\

There are two possible ways of coregistrating the EEG/MEG data into the structural MRI space. 

\begin{enumerate}
    \item A Landmark based coregistration (using fiducials only).\\
    The rigid transformation matrices (Rotation and Translation) are computed such that 
they match each fiducial in the EEG/MEG space into the corresponding one in sMRI space. The same 
transformation is then applied to the sensor positions.
    \item Surface matching (between some headshape in MEG/EEG space and some sMRI derived scalp tesselation).
For EEG, the sensor locations can be used instead of the headshape. For MEG, the headshape is 
first coregistrated into sMRI space; the inverse transformation is then applied to the head model
and the mesh.\\
Surface matching is performed using an Iterative Closest Point algorithm (ICP). 
The ICP algorithm~\cite{Besl_McKay} is an iterative alignment algorithm that works in three phases:
\begin{itemize}
    \item Establish correspondence between pairs of features in the two structures that 
    are to be aligned based on proximity;
    \item Estimate the rigid transformation that best maps the first member of the 
     pair onto the second;
    \item Apply that transformation to all features in the first structure. These three
     steps are then reapplied until convergence is concluded.
Although simple, the algorithm works quite effectively when given a good initial estimate.
\end{itemize}
\end{enumerate}

In practice what you will need to do after pressing  the 'Coregister' button is to specify the points
in the sMRI image that correspond to your M/EEG fiducials. If you have more than fiducials (which may happen
for EEG as in principle any electrode can be used as a fiducial), you will be ask at the first step to select
the fiducials you want to use. You can select more than 3, but not less. Then for each M/EEG fiducial you selected
you will be asked to specify the corresponding position in sMRI image in one of 3 ways. 

\begin{itemize}
\item 'select' - locations of some points such as the commonly used nasion and preauricular points and also
CTF recommended fiducials for MEG (as used at the FIL) are hard-coded in SPM. If your fiducial corresponds to one of these
points you can select this option and then select the correct point from a list.
\item 'type' - here you can enter the MNI coordinates for your fiducial ($1 \times 3$ vector). If your fiducial is not
on SPM's hard-coded list, it is advised to carefully find the right point on either the template image or
on your subject's own image normalized to the template. You can do it by just opening the image using SPM's
Display/images functionality. You can then record the MNI coordinates and use them in all coregistrations you
need to do using the 'type' option.
\item 'click' - here you will be presented with a structural image where you can click on the right point. 
This option is good for 'quick and dirty' coregistration or to try out different options. 
\end{itemize}

You will also have the option to skip the current fiducial, but remember you can only do it if you eventually
specify more than 3 fiducials in total. Otherwise the coregistration will fail. 

After you specify the fiducials you will be asked whether to use the headshape points if they are available. 
For EEG it is advised to always answer 'yes'. For MEG if you use a head model based on the subject's sMRI and have
a precise information about the 3 fiducials (for instance by doing a scan with fiducials marked by vitamin E capsules)
using the headshape might actually do more harm than good. In other cases, it will probably help as in EEG. 

The results of coregistration will be presented in SPM's graphics window. It is important to examine the results
carefully before proceeding. In the top plot you will see the scalp, the inner skull and the cortical mesh
with the sensors and the fiducials. For EEG make sure that the sensors are on the scalp surface. For MEG check
that the head positon in relation to the sensors makes sense and the head does not for instance stick outside
the sensor array. In the bottom plot the sensor labels will be shown in topographical array. Check that the top
labels correspond to anterior sensors, bottom to posterior, left to left and right to right and also that the labels
are where you would expect them to be topographically.   

\section{Forward computation (\textit{forward})}
This refers to computing for each of the dipoles on the cortical mesh the effect it would have on the sensors. The result
is a $N \times M$ matrix where N is the number of sensors and M is the number of mesh vertices (that you chose from several options
at a previous step). This matrix can be quite big and it is, therefore, not stored in the header, but in a separate *.mat file which has
'SPMgainmatrix' in its name and is written in the same directory with the dataset. Each column in this matrix is a so called 'lead field'
corresponding to one mesh vertex. The lead fields are computed using the 'forwinv' toolbox developed by Robert Oostenveld, which 
SPM shares with Fieldtrip. This computation is based on physics theory and it needs some assumptions about the physical properties of the head.
There are different ways to specify these assumptions which are known as 'forward models'. 'forwinv' toolbox can support different kinds of forward 
models. In SPM8 beta we only use a one-shell boundary element model (BEM) for MEG and 3-shell BEM for EEG. In the future we will introduce additional
options. So in SPM8 beta when you press 'Forward Model' button (which should be enabled after successful coregistration), you will not be asked
any additional questions. The lead field matrix will be computed and saved. This is a time-consuming step and it takes longer for high-resolution
meshes.


\section{Inverse reconstruction}
To get started press the 'Invert' button. The first
choice you will see is between 'Classical', 'VB-ECD' and 'DCM'. DCM which is explained in detail elsewhere can also be seen as an 
informed source reconstruction approach. It can, therefore be invoked at this point. We will not go into further details here as you
can find them in chapter \ref{Chap:eeg:DCM}. For reconstruction based on an empirical Bayesian approach to localize either the evoked response, 
the evoked power or the induced power, as measured by EEG or MEG press the 'Classical' button. If you have trials belonging to more than one condition
in your dataset then the next choice you will have is whether to invert all the conditions together or choose a subset. It is recommended to invert
the conditions together if you are planning to later do a statistical comparison between them. If you have only one condition or after choosing
the conditions you will get a choice between 'Standard' and 'Custom' inversion. If you choose 'Standard' inversion SPM will start the computation
with default settings which are using the multiple sparse priors (MSP) algorithm ~\cite{karl_msp} on the whole data segment that is in the input.
If you want to fine-tune the parameters of the inversion, choose the 'Custom' option.
Then you will have the possibility to choose between several types of inversions differing by their hyperprior models (IID - equivalent to classical
minimum norm, COH - smoothness prior similar to methods such as LORETA) or optimization scheme for the MSP method (GS - greedy search, ARD - automatic
relevance determination). You can the choose the time window that will be available for inversion. Generally, it is not recommended to limit the time
segment you use for inversion just to what you are interested in, unless there are stimulus artefacts
that need to be excluded. It is better to let the algorithm model all the brain sources generating the response and then to focus on the sources
of interest using the appropriate contrast (see below). There is also option to apply hanning taper to the channel time series in order to downweight
the possible baseline noise at the beginning and end of the trial and an option to pre-filter the data. Finally you can restrict the solutions to particular
brain areas by loading a *.mat file with a $K \times 3$ matrix containing MNI coordinates of the areas of interest. This option seems strange at the first
sight as it looks like providing the source reconstructions with the answers you expect to get from it. The idea is that in the Bayesian inversion
framework you have the possibility to compare inversions with different settings applied to the same data using Bayesian model comparison. By limiting
the solutions to particular brain areas you greatly simplify your model and if that simplification really captures the sources generating the response
the restricted model will have much higher model evidence than the unrestricted one. If, however, the sources you suggested cannot account for the data,
the restriction will result in worse model fit and depending on how much worse it is, the unrestricted model might be better in the comparison. So using
this option with subsequent model comparison is a way, for instance, to integrate prior knowledge from the literature of from fMRI/PET/DTI into your
inversion and also compare between alternative prior models. Note that for model comparison to be valid all the settings that affect the input data,
like time window, conditions used and filtering should be the same.

Once the inversion is completed you will see in the graphics window at the top plot the time course of the region with the highest activity and
at the bottom plot the maximal intensity projection (MIP) at the time of the highest activation. You will also see the log-evidence value that can
be used for model comparison as explained above. Note that not all the output of the inversion is displayed. The full output consists of time courses
for all the sources and conditions for the entire time window. You can view more of the results using the controls at the bottom right corner
of the 3D GUI. That allow focusing on a particular time, brain area and condition and also displaying a movie of the evolution of activity in time.

\section{Summarizing the results of inverse reconstruction as an image}
If you want to do more than eyeball your results in silent appreciation, SPM offers the possibility to proceed by summarizing some aspects
as of the inversion results as 3D NIfTI images and proceeding with GLM-based statistical analysis using random field theory based thresholding similar
to the 2nd level analysis in fMRI to make inferences about region and trial-specific effects (at the between subject level).
This entails summarizing the trial and subject specific response with a single 3-D 
image in source space. Critically this involves prompting for a time-frequency contrast window to create each contrast image.
This is a flexible and generic way of specifying the data feature you want to make an inference about (e.g., gamma activity around 300 
ms or average response between 80 and 120 ms). This kind of contrasts is specified by pressing the 'Window' button. You will then be asked about
the time window of interest (in ms, peri-stimulus time). It is possible to specify either a single time value or a time segment. The next question is about the frequency band.
If you just want to average the source timecourse leave that at the default, zero. In this case the window will
be weighted by a gaussian. In case of single time point this will be a gaussian with 8 ms full width half maximum (FWHM). 
If you specify a particular frequency or a frequency band, then a series of Morlet wavelet projectors
will be generated summarizing the energy in the time window and the band of interest. There is a difference between specifying frequency band of intereste as zero as opposed
to specifying a wide band that covers the whole frequency range of your data. In the former case the time course of each dipole will be averaged weighted by a gaussian. Therefore,
if within your time window this time course changes polarity, the activity can average out and in an ideal case even a strong response can produce a value of zero. In the latter case
the power is integrated over the whole spectrum ignoring phase, and this would be equivalent to computing the sum of squared amplitudes in the time domain. 
Finally, you will have a choice between 'evoked' and 'induced'. Here comes into play
the possibility we only briefly mentioned before to load an epoched rather than averaged filed for the inversion. If you have multiple trials for a certain conditions
the projectors generated at the previous step can either be applied to each trial and the results averaged (induced) or applied to the averaged trials (evoked). Thus it is
possible to perform localization of induced activity that has no phase-locking to the stimulus. It is also possible to focus on frequency content of the ERP using the 'evoked'
option but clearly the results will not be the same. The projectors you specified (bottom plot) and the resulting MIP (top plot) will be displayed when the operation is completed.

At this stage you have the possibility to export your image as NIfTI by pressing the 'Image' button. You will be asked to specify the amount of smoothing. An unsmoothed image will
be exported in any case. Smoothing can help to get better results in an accross subject statistical test. However, it compromizes your spatial resolution. You should try different options
to see what works for your data. When you use the group inversion (see below). You might not need smoothing at all. 

Note that when a file contains several conditions the images are normalized so that sum of power across all conditions is 1. Thus you might get differently scaled images if you invert each
condition alone as opposed to inverting them together. It is recommended to put all the conditions you intend to compare in the same inversion. 

\section{Rendering interface}
By pressing the 'render' button you can open a new GUI window which will show you a rendering of the inversion results on the brain surface. You can rotate the brain, focus on different
time points, run a movie and compare the predicted and observed scalp topographies and time series. A useful option is 'virtual electrode' which allows you to extraxt the time course from any
point on the mesh and the MIP at the time of maximal activation at this point. Just press the button and click anywhere in the brain.\\
An additional tool for reviewing the results of SPM source reconstructions is available in the SPM M/EEG reviewing function.

\section{Group inversion}
One problem that we encountered with MSP inversion is that sometimes it was 'too good' producing solutions that were so focal in each subject that the spatial overlap between the activated
areas accross subjects was not sufficient to yield a significant result in across-subjects contrast. This could be improved by smoothing, but smoothing compromises the spatial resolution
and thus subverts the main advantage of using an inversion method that can produce focal solutions. To circumvent this problem we proposed a modification of the MSP method ~/cite{vl\_group}
that effectively restricts the activated sources to be the same in all subjects with only the degree of activation allowed to vary. We showed that this modification makes it possible to 
obtain significance levels close to those of non-focal methods such as minimum norm while preserving accurate spatial localization. The group inversion can yield much better results
than individual inversions because it introduces an additional constraint for the ill-posed inverse problem, namely that the responses in all subjects should be explained 
by the same set of sources. Thus it should be your method of choice when analyzing an entire study with subsequent GLM analysis of the images. Group inversion works very similarly to
what was described above. You can start it by pressing the 'Group inversion' button right after opening the 3D GUI. You will be asked to specify a list of M/EEG datasets to invert together.
Then the routine will ask you to perform coregistration for each of the files and specify all the inversion parameters in advance. It is also possible to specify the contrast parameters in advance.
Then the inversion will proceed by computing the inverse solution for all the files and writing out the output images. The results for each subject will also be saved in the header 
of the corresponding input file and it is possible to load this file into the 3D GUI after the inversion and explore the results as described above. 

\section{Appendix: Data structure}
The Matlab object describing a given EEG/MEG dataset in SPM is denoted as \textit{D}. 
Within that structure, each new inverse analysis will be described by a new cell of sub-structure 
field \textit{D.inv} and will be made of the following fields:

\begin{itemize}
    \item \textit{method}: character string indicating the method, either 'ECD' or 'Imaging' in present case;
    \item \textit{mesh}: sub-structure with relevant variables and filenames for source space and head modeling;
    \item \textit{datareg}: sub-structure with relevant variables and filenames for EEG/MEG data registration into MRI space;
    \item \textit{forward}: sub-structure with relevant variables and filenames for forward computation;
    \item \textit{inverse}: sub-structure with relevant variable, filenames as well as results files;
    \item \textit{comment}: character string provided by the user to characterize the present analysis;
    \item \textit{date}: date of the last modification made to this analysis.
\end{itemize}

