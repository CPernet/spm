\chapter{SPM for MEG/EEG overview \label{Chap:eeg:overview}}

Like the previous SPM5 version, SPM8 provides for the analysis of EEG
and MEG data. The SPM8 software is much more robust than the previous
version, in many aspects like conversion of data, source
reconstruction, and dynamic causal modelling.
\\
\\
For three years, we have collected valuable experience
for the analysis of M/EEG data, and received a lot of valuable
feedback from both FIL and external collaborators and users. We had
plenty of opportunity to see which things worked well and what can be
improved. One of our major insights was that writing a general routine
for conversion of M/EEG data from their native to our SPM-format is a
major effort. This is simply because there are so many different
formats 
around and it is quite an undertaking for a small developer team like
ours to write stable software which can read all formats, some of
which we have never seen ourselves. This had two consequences. The
first is that we now collaborate with the developers of fieldtrip
(head developer: Robert Oostenveld, F.C. Donders centre in
Nijmegen/Netherlands). Robert already had made available a wide range of
matlab code to convert M/EEG data, and we thought it a good idea to
let both SPM and fieldtrip use and develop the same code. The second
major change is that we changed the internal M/EEG format of SPM in
many ways to make reading/writing and manipulating M/EEG data more
robust and straightforward for the user. Effectively, we ripped out
the old M/EEG engine from SPM5, and put in a much more powerful one.
\\
\\
There are a couple of other major changes from SPM5 to SPM8. 

First, based on our work about source reconstruction, we have
implemented a number of routines which provide for a robust and
efficient source reconstruction, using Bayesian approaches. The
resulting, voxel-based source reconstructions can then be analysed, at
the group level, with the same well-tested routines which one uses for
fMRI data.

Second, Dynamic Causal Modelling, a network analysis for spatiotemporal
M/EEG data, has been developed further over the past three years. The
DCM routines for modelling evoked responses or 
fields have been significantly improved both in functionality and
speed. There are various exciting extensions to DCM for M/EEG, e.g. we
now provide for models of induced responses and local field
potentials. 

Third, there are now three ways of how users can control the
SPM for M/EEG software. One can use the graphical user interface, or two
different possibilities of using scripts to batch jobs. These batch
facilities come in handy for multi-subject studies. Like in fMRI
analysis, many processing steps are repetetive and it is now quite
straightforward to automize the software to a high degree.

Fourth, it is now possible to convert, in working memory, SPM
data to fieldtrip-data, and back. This feature makes it possible to 
use, within SPM, many fieldtrip functions. It would be quite
straightforward, using a script, to work within SPM, but use fieldtrip
functions to do parts of the preprocessing. 
\\
\\
The following chapters will go through all the EEG/MEG related
functionality of SPM8. All users will probably find the tutorial
useful for a quick start. A further detailed description of the
conversion, preprocessing functions, and the display is given in chapter
\ref{Chap:eeg:preprocessing}. The 3D-source reconstruction routines,
including few-dipole models, are described in chapter
\ref{Chap:eeg:reconstruction}. In chapter \ref{Chap:eeg:stats}, we
explain how one would use the SPM's statistical machinery to analyse
M/EEG data. Finally, in chapter \ref{Chap:eeg:DCM}, we describe the
graphical user interface for dynamical causal modelling, for evoked  
responses, induced responses, and local field potentials.

