\chapter{SPM for MEG/EEG overview \label{Chap:eeg:overview}}

\section{Welcome to SPM for M/EEG}
With SPM8 you can analyze all kind of MEG
and EEG data. Our group has been highly productive in the recent years
in publishing peer-reviewed papers about M/EEG analysis, in particular
about source reconstruction and Dynamic Causal Modelling (DCM), which
is a spatiotemporal network model to estimate effective connectivity
in a network of sources. All these new methods, mostly based on the
Bayesian approach, have been coded up and put into SPM8. We
provide for the full analysis flow, i.e., you can take your raw data
from the MEG or EEG machine, and put it through SPM, starting from the
conversion of the data through to a statistical analysis of source
reconstructed multi-subject data or Dynamic Causal Modelling.
\\
\\
Our overall goal is to provide an academic M/EEG analysis software
package that can be used by everyone to use the most recent and
sophisticated methods available for the analysis of M/EEG data. As you
may guess, this goal is quite ambitious because there is a large
number of different M/EEG formats available, plus there are literally
dozens of different analysis strategies that researchers would like to
use. Clearly, our rather small group doesn't have the resources
to cover all these different approaches. However, we made SPM for
M/EEG as open as it possibly can be to allow researchers to use
their favourite analysis software for specific processing steps. For
example, it is possible to convert data to SPM8, then convert the data
to Fieldtrip or EEGlab (using an SPM conversion routine), use a couple
of functions in these packages, convert back to SPM, and do source
reconstruction or DCM. Any combination of processing steps should be
possible, and we expect that this 
software-interoperability among analysis software packages (each with
its own area of expertise) will lead to a boost of M/EEG researchers
trying out new ways of analysing their data with a wide range of
sophisticated methods. We are pleased to say that we have a formal
collaboration with the excellent Fieldtrip package (head developer
Robert Oostenveld) on many analysis issues. For example, SPM and
Fieldtrip share routines for converting data to matlab, forward modelling 
for M/EEG source reconstruction and the SPM8 distribution contains
a version of Fieldtrip so that you can combine Fieldtrip and SPM 
functions in your custom scripts. SPM and Fieldtrip complement each
other well as SPM is geared toward very specific analysis tools as 
will be described below whereas Fieldtrip is a more general repository 
of different methods that can be put together in flexible ways
to perform a variety of analyses. This flexibility of Fieldtrip, however, 
comes at the expense of accessibility to a non-expert user. Fieldtrip does
not have graphical user interface (GUI) and its functions are used by 
writing custom Matlab scripts. In combined SPM8-Fieldtrip the
flexibility of Fieldtrip can be complemented by the SPM GUI 
tools that allow rapid development of simple GUI's and also SPM8's new
powerful batching system. Within this framework the power users in a lab
can easily and rapidly develop specialized analysis tools with GUI that 
can then  also be used by non-proficient Matlab users. Some examples of such
tools are available in MEEG toolbox distributed with SPM. We will also be
glad to include in this toolbox new tools contributed by the users as long
as they are of general interest and applicability. 
\\
\\
SPM's speciality is, of course, the statistical analysis of voxel-based
images. For the statistics, we use exactly the same routines as SPM
for fMRI users would do. These are well-tried and trusted functions
based on the general linear model and Gaussian random field theory.
These routines have been developed and used in the fMRI field over
many years and are equally applicable to multi- (or single-) subject
M/EEG studies.
\\
\\
Furthermore, our group has invested heavily in establishing Bayesian
approaches to the source reconstruction of M/EEG data. Good source
reconstruction techniques are vital for the M/EEG field, otherwise it
would be very difficult to relate sensor data to neuroanatomy or
findings from other modalities like fMRI. Our belief is that source
reconstruction should be done in a Bayesian way to allow for a
principled way of incorporating prior beliefs about how the data were
generated, and to enable proper model comparison. With the use of
priors and Bayesian model comparison, M/EEG source reconstruction is a
very potent neuroimaging tool, which has a unique macroscopic view on
neuronal dynamics.

In addition, we have taken the idea of Dynamic Causal Modelling (DCM)
from the fMRI domain, and put it to some good use in the M/EEG field. For
M/EEG, DCM is a powerful technique, because the data are highly
resolved in time and make the identifiability of neurobiologically
inspired network models feasible. This means that DCM can make
inference about temporal precedence of sources and quantify
feedforward, backward and lateral connectivity among sources on a
neuronal time-scale of milliseonds. Note that DCM/fMRI won't do this
for you; DCM/fRMI (or any other connectivity analysis in fMRI)
look at the rather slow expression of a modulation of connectivity by
task. These happen on a time-scale of seconds. With M/EEG, you make
inference about dynamics at a time-scale of milli-seconds, which seems
a bit more exciting.

\section{Changes from SPM5 to SPM8}
Like the previous SPM5 version, SPM8 provides for the analysis of EEG
and MEG data. The SPM8 software is much more robust than the previous
version, in many aspects like conversion of data, source
reconstruction, and dynamic causal modelling.
\\
\\
For three years, we have collected valuable experience
for the analysis of M/EEG data, and received a lot of valuable
feedback from both FIL and external collaborators and users. We had
plenty of opportunity to see which things worked well and what can be
improved. One of our major insights was that writing a general routine
for conversion of M/EEG data from their native to our SPM-format is a
major effort. This is simply because there are so many different
formats 
around and it is quite an undertaking for a small developer team like
ours to write stable software which can read all formats, some of
which we have never seen ourselves. This had two consequences. The
first is that we now collaborate with the developers of Fieldtrip
(head developer: Robert Oostenveld, F.C. Donders centre in
Nijmegen/Netherlands). Robert already had made available a wide range of
matlab code to convert M/EEG data, and we thought it a good idea to
let both SPM and Fieldtrip use and develop the same code. The second
major change is that we changed the internal M/EEG format of SPM in
many ways to make reading/writing and manipulating M/EEG data more
robust and straightforward for the user. Effectively, we ripped out
the old M/EEG engine from SPM5, and put in a much more powerful one.
\\
\\
There are a couple of other major changes from SPM5 to SPM8. 

First, based on our work about source reconstruction, we have
implemented a number of new routines which provide for a robust and
efficient source reconstruction, using Bayesian approaches. The
resulting, voxel-based source reconstructions can then be analysed, at
the group level, with the same well-tested routines which one uses for
fMRI data.

Second, Dynamic Causal Modelling, a network analysis for spatiotemporal
M/EEG data, has been developed further over the past three years. The
DCM routines for modelling evoked responses or 
fields have been significantly improved both in functionality and
speed. There are now various exciting extensions to DCM for M/EEG,
e.g. we now provide for models of induced responses and local field
potentials. 

Third, there are now three ways of how one can control the
SPM for M/EEG software. There are the graphical user interface, and two
different possibilities of using scripts to batch jobs. These batch
facilities come in handy for multi-subject studies. Like in fMRI
analysis, many processing steps are repetetive and it is now quite
straightforward to automize the software to a high degree.

Fourth, it is now possible to convert, in working memory, SPM
data to Fieldtrip or EEGlab, and back. This feature makes it possible to 
use, within SPM, many Fieldtrip and EEGlab functions. For example, it
is quite straightforward, using a script, to work within SPM, and use
Fieldtrip functions to do parts of the preprocessing. 
\\
\\
The following chapters will go through all the EEG/MEG related
functionality of SPM8. All users will probably find the tutorial
useful for a quick start. A further detailed description of the
conversion, preprocessing functions, and the display is given in chapter
\ref{Chap:eeg:preprocessing}. The 3D-source reconstruction routines,
including few-dipole models, are described in chapter
\ref{Chap:eeg:reconstruction}. In chapter \ref{Chap:eeg:stats}, we
explain how one would use the SPM's statistical machinery to analyse
M/EEG data. Finally, in chapter \ref{Chap:eeg:DCM}, we describe the
graphical user interface for dynamical causal modelling, for evoked  
responses, induced responses, and local field potentials.

