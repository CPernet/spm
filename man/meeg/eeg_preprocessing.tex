\chapter{EEG/MEG preprocessing --- Reference
  \label{Chap:eeg:preprocessing}} 

In this chapter we will describe the function and syntax of all 
SPM/MEEG preprocessing and display functions. This will be the most
detailed description of the functions  in this
manual. Our goal is to provide a comprehensive description of how the
software can be used to preprocess M/EEG data up to the point where
one would use one the source reconstruction techniques or statistical
analysis of M/EEG channel data.

These functions can be called either from SPM's graphical user
interface (GUI), from the matlab command line, or via the batch input
system. We recommend beginners to use the GUI first, because this will
prompt SPM to ask all relevant information which are needed to process
the data. The batch input system is meant to cover repetitive analyses
of data once the user knows what should be done, in which order. The
command line facilities are very useful for writing scripts, or using
SPM's history-to-script functionality to generate scripts
automatically. The command line use of SPM for M/EEG will require some
matlab knowledge.

For the command line, we follow the concept of providing only one
input argument to each function. This input argument is usually a 
structure (struct) that contains all input arguments as fields. This
approach has the advantage that the input does not need to follow a
specific input argument order. If an obligatory input argument is
missing, the function will invoke the GUI and ask the user for the
missing argument. When using the GUI, a function is called without any
input argument, i.e.~SPM will ask for all input arguments. If using
the command line (e.g., with a script), you can specify all arguments
in advance and effectively use SPM/MEEG functions in batch mode. We
provided some sample batch script (\textit{meeg\_preprocess}) in the
\textit{man/example\_scripts/} folder of the distribution.

\section{Conversion of data}
The first step of any analysis is the conversion of data from its
native machine-dependent format to a matlab-based, common SPM
format. This format stores the data in a *.dat file and all other
information in a *.mat file. The *.mat file contains the data
structure D and the *.dat is the M/EEG data. The conversion facility
of SPM is based on the "fileio" toolbox
(http://www2.ru.nl/fcdonders/fieldtrip/doku.php?id=fieldtrip:development:fileio),
which is shared between SPM8, Fieldtrip and EEGLAB toolboxes and
jointly developed by the users of these toolboxes. At the moment most
common EEG and MEG data formats are supported. For some cases, it
might be necessary to install additional Matlab toolboxes. In this
case an error message will be displayed with a link where the
appropriate toolbox can be downloaded.

In the following, we will first describe the GUI-version for
conversion, and then how one would convert data using the batch or a
script. After clicking on the Convert button of the M/EEG GUI you will
be asked to select the file to be converted. As a rule of thumb, if the
dataset consists of several files, the file containing the data (which
is usually the largest) should be selected. SPM can usually
automatically recognize the data format and apply the appropriate
conversion routine. However, in some cases there is not enough
information in the data file for SPM to recognize the format. This
will typically be the case for files with non-specific extensions
(*.dat, *.bin, *.eeg etc.). In these cases the header-, and not the
data-, file should be chosen for conversion and if it is recognized,
SPM will locate the data file automatically. Note that SPM8 can also
convert data in SPM5 format, where the the *.mat file should be
selected.

After the file is chosen, you will be asked (Define settings?) to
choose whether to define some settings for the conversion or 'just
read'.  The latter option was introduced to enable a simple and
convenient conversion of the data with no questions asked. The
resulting SPM M/EEG data file can then be explored with SPM tools to
determine the appropriate conversion parameters for the future. If
the 'just read' option is chosen, SPM will try to convert the whole
dataset preserving as much data as possible. The other option - 'yes'
let you control all features of the conversion to convert only
the data that will be used in subsequent processing.

If this option is chosen, the next question will be whether to read
the data as 'continuous' or as 'trials'. Note that some datasets do
not contain continuous data to begin with. These datasets can only be
converted with the 'trials' option.  

If the 'continuous' option is chosen you will be asked (Read
everything?) whether to convert the whole file (yes) or a subset of it
(no). If the answer is 'no' you will be asked to specify the time
window in seconds. Note that if a data file contains several
concatenated long segments (e.g., if the recording was paused
and then resumed) only one of these segments at a time can be
converted as continuous.  Therefore you should specify a time window
which does not cross the boundaries between segments.  

If the 'trial' option is selected, the next question will be where to
retrieve the information about trials. There are three options. If
'data' is chosen, SPM will attempt to look for information about
trials in the 
dataset. This option is suitable for datasets that are already epoched
or datasets which contain some information about trials. If 'define'
is selected, trials can be defined based on information about events
which appears in the file. This routine used for this option is
identical to the one used in epoching (see below) and the resulting
SPM file will be already epoched. The advantage of defining trials at
conversion is that only the necessary subset of the raw data is
converted. This is useful when the trials of interest are only a small
subset of the whole recording. After the trial definition is
completed, the results can be saved into a file. This file can be
later used instead of repeating the trial definition again. This is
done by selecting the third trial definition option - 'file'. 

The next question will be about which channels should be
converted. Five options are available. 'all' - convert 
all the available channels. 'meg', 'eeg' - automatically detect and
choose MEG and EEG channels respectively. Currently, these options may
not work correctly for some MEG and EEG systems. (This is because many
data formats do not provide information about what data were acquired
by a specific channel. We are working on automatic ways to guess the
right channel type anyway.) 'gui' - choose the channels to
convert using a graphical interface. The overall selection of channels
can be saved in a file and this file can later be used by choosing the
'file' option.

The final question is by which name the new SPM EEG files should be
written to disk. By default SPM will add the prefix 'spm8\_' to the
name of the raw data file if the data is read as continuous and
'espm8\_' if the data is read by trials.  

SPM will now convert the data. This may take some time depending on
file size and a red bar will inform you about the progress.

\section{Integration of SPM and Fieldtrip}
The SPM8 distribution includes the latest version of 'Fieldtrip' toolbox
(http://www2.ru.nl/fcdonders/fieldtrip/). FieldTrip is a Matlab
toolbox for MEG and EEG analysis that is being developed at the
F.C. Donders Centre (FCDC) together with collaborating
institutes. Fieldtrip functions can be used for many kinds of analysis
which are currently not supported in SPM proper. However, Fieldtrip
does not have extensive graphical user interface and its functionality
should be accessed by writing scripts. Full reference documentation
for Fieldtrip including example scripts is available at the Fieldtrip
website (http://www.ru.nl/neuroimaging/fieldtrip/). The SPM
distribution also contains some documentation, contained as help
comments in Fieldtrip functions. These can be found in the directory 
external/fieldtrip/private. Note that in order to prevent function
name clashes SPM calls its Fieldtrip functions via 
intermediate or 'wrapper' functions whose name always starts with
'ft\_'. Therefore, to adapt a standard Fieldtrip script for use with SPM
you must add this prefix to all Fieldtrip function names in the
script.

Fieldtrip data structures can be converted to SPM EEG files using
the \textit{spm\_eeg\_ft2spm} function.  SPM M/EEG data, once loaded
with 
the function \textit{spm\_eeg\_load} can be converted to Fieldtrip
format using the method 'ftraw' (with syntax D.ftraw or ftraw(D)).


\section{The M/EEG SPM format}
The M/EEG format has changed from SPM5 to SPM8. There were many
reasons why we decided that it was time to radically change the
format. If you still have some data from your SPM5 analyses, you can
convert them to SPM8 (see above). 

Here some explanation why we changed the format. Previously, we placed
all header information into a struct, which was then stored in a
mat-file. When the data were read, the struct-file was put into
working memory, and the data, contained in the dat-file, was linked to 
the struct, using memory-mapping. We found that making a struct
available in working memory was highly problematic. This was because
users would manipulate the 
struct but sometimes introduced an error to the format (e.g., they
removed a few 
trials from the data but did not update all fields relating to the
total number of trials). This would generate hard-to-resolve errors
when trying to further process the now inconsistent data. To avoid
this in the future, we introduced two changes. The first is that
SPM8 now 
always does a consistency check when saving a file. This means, if,
for whatever reason, the data format was made inconsistent, SPM8 will
now report this as soon the user tries to write these data. SPM8 will
also report where the check flagged an inconsistency. Second,
when reading the data, we now convert the header struct to an object
and only then make it available in working memory. Generally, this
matlab object can only be manipulated using standardized functions
(called methods), which makes it very hard to introduce any
inconsistency into SPM M/EEG data in the first place. Also, using
methods simplifies internal book-keeping, which makes it much easier
to program functions operating on the M/EEG object. For example, while
the SPM5 format kept a variable which contained the number of total
trials, the SPM8 object does not have this variable but a method that
returns the number of trials by simply counting how many trials of
data there are. This makes it easy for a programmer to, for example,
remove trials. There is no more need to update a number of trials
variable. There are many more simplications along these lines due to
the object and methods. In summary, the change in format resulted in a
much more robust, and usable data format.

\section{Preparing the data after conversion}
Although SPM tries to do its best to extract information automatically
from the various data formats, some information is not available in
the original files and needs to be supplied by the user. Usually,
these information are things like which channel is MEG, EEG, or
measured other things like stimulus triggers. Also, what is usually
not in original EEG files, are information about channel locations, as
for example measured by a digitizer system like Polhemus. 
Reading and linking this additional information with the data is the
purpose of the 'prep' interface. This interface is accessed by
selecting 'prep' from the 'Other' drop-down menu in the GUI.  A menu
(easily overlooked) will appear at the top of SPM's interactive 
window.

In this menu, an SPM M/EEG file can be loaded and saved using the 'File'
submenu. 'Channel types' submenu allows reviewing and
changing the channel types. Use the 'Review' option to examine the
presently set channel types. During conversion, SPM tries to do its
best to 'guess' the correct channel types but this can sometimes go
wrong, in particular for EEG data. To set a particular channel group
to some channel type, select this type from the menu. A list of all
channels will appear. Select the subset whose type you would like to
set. Ctrl and Shift buttons can be used to refine the selection. Press
OK to apply your choice. It is especially important to correctly specify
which are the EEG channels. MEG type is assigned automatically by SPM
and cannot be modified in GUI.  

The 'Sensors' submenu can be used to supply information about the
sensor positions to the file. Sensor positions for MEG are extracted
from the raw data automatically and are already present. For EEG,
sensor positions are usually measured by a special device (such as
Polhemus) and are not part of the dataset. EEG sensor positions can be
added to an SPM file using 'Load EEG sensors' menu. There are 3
options:

\begin{itemize}
\item	loading from a *.mat file: the file should contain an $N
  \times 3$ matrix, where $N$ is the same as the number of channels
  whose type is set to 'EEG' and the order of the rows matches the
  order of these channels in the SPM file.   

\item	loading from a FIL Polhemus file. This option is suitable for
  loading files generate by the Polhemus device used at the FIL. Note
  that if you use a different setup than we do, it is highly unlikely
  that your files will be suitable for loading using this
  option. Polhemus is just a name of a piece of hardware and Polhemus
  files are generated by different software utilities that come with
  it and their formats vary between sites. To read your channel
  locations you should convert your measured locations-file to one of
  the  formats supported by the next option.  

\item	'Convert locations file' - this option uses a function from
  the internal fileio toolbox that supports several common formats for
  EEG channel position specification such as *.sfp and
  BESA's *.elp. An *.sfp file is an ASCII file containing a column of
  channel labels and 3 columns of cartesian coordinates. Check the
  fieldtrip documentation, or the help-text for fieldtrip's function
  \textit{read\_sens} for a complete list of supported formats. Within SPM,
  you find this function hidden away under
  \textit{external/fileio/private/read\_sens.m}.  

\end{itemize}

In order for SPM to make sense of the read sensor positions you will
also need to load a set of fiducial points which are specified in the
same 
coordinate system. The most commonly used fiducials are the nose
bridge and the two pre-auricular points. The fiducials can be loaded
using the 'Load Fiducials/Headshape' menu. As in the case of sensors,
it is possible to load fiducials from a *.mat file, that should
contain a $K \times 3$ matrix, where $K$ (usually 3) is the number of
fiducials. An additional option is to convert a headshape file. A
headshape file is usually generated by the operator moving his
digitizer pen around on the subject's scalp and contains many more data 
points than just 3 fiducials. This option should be used for all other
supported file types including FIL Polhemus files and EEG sensor
position files. Even if you do not
measure electrode positions routinely in your lab, we recommend to
perform at least one initial measurement with the electrode cap you
use and use the result as a standard template. If the sensor positions
and fiducials are in the same file (e.g. *.sfp or FIL Polhemus) this
file should be loaded twice - as sensor positions file and as
fiducials /headshape file. Of course, you can also load
fiducials/headshape file for MEG data.

Alternatively, you can also proceed without using measured fiducials
or sensor locations but with a standard template. This is supplied by
SPM automatically based on the data format and some assumptions
derived from channel labels in the converted file. These templates
provide you with a complete set of pre-specified 3D- and
2D-coordinates, including fiducials.

If you do not use the standard templates, after loading the sensor
positions you can perform coregistration of your sensors with SPM's
template head model. This initial alignment is helpful to verify that
the sensor information you supplied were interpreted correctly and
should also be done if you would like to generate a 2D sensor template
based on your 3D sensor positions (see below). The 2D-coordinates will
be used for displaying the data in a topologically meaningful way. This
is done using the 'Coregister' option.

The '2D Projection' menu deals with the generation of representative
2D-coordinates for the sensors. Note that generating 2D-coordinates is
not 
obligatory. If the 2D-coordinates are not specified, the sensors will
be, when displaying, presented in a default square grid. Missing out
on topographically meaningful 2D-coordinates might be useful when
working on few channels. The 2D-coordinates are also used for
producing scalp-level SPMs in voxel space when converting M/EEG data
to images for later statistical analysis (see below). If you are 
planning to do 3D source reconstruction or 
DCM, 2D-coordinates are not necessarily required. Also, you can load
2D-coordinates from a file (several example files are available in
the \textit{EEGtemplates} SPM directory). For EEG, 2D-coordinates can also
be generated by projecting the 3D sensor positions to a plane. For
this to work coregistration should be performed first (see
above). The resulting 2D-coordinates are displayed in SPM's graphics 
window. You can modify these projected 2D-coordinates
manually by adding, deleting and moving sensors. To select a sensor,
click on its label. The label will change its color to green. If
you then click at a different location, the sensor will be
moved to this position. Note that, at this stage, SPM does not
check whether there is any correspondence between the labels of the
coordinates and the labels of the channels stored in the SPM
file. When you are satisfied with the 2D-coordinates, select 'Apply'  
from the menu and the coordinates will be assigned to EEG or MEG
channels according to their labels. Note that 2D-coordinates cannot 
be assigned to channels of other types than M/EEG.  

Remember to save the file using File/Save after you finished modifying
it using the prep interface. Your changes will not be saved
automatically.

In the rare case that you neither have measured sensor locations,
or fiducials, and the supplied standard templates do not work for you,
you can also supply a so-called channel template file, which contains
all information necessary. However, remember, that if you do not
supply any 2D-coordinates, you can still use all SPM functions,
however, SPM will use 2D-coordinates laid out in a topographically not 
meaningful rectangular pattern.

A channel template file contains four variables:\\

\begin{tabular}{llcp{9cm}}
{\bf Nchannels} & &  - & The number of channels\\
{\bf Cnames}&  & - & A cell vector of channel names. Each cell can
contain either a string or a cell vector of strings. The latter allows
to have multiple versions of a given channel name. Case can be
ignored, i.e.~it doesn't matter whether channel names are in small or
capital letters.\\
{\bf Cpos} & & - & A $2 \times Nchannels$-matrix of channel
coordinates on a 2D plane. In $x$- and $y$-direction the minimum
coordinate must be $\leq 0.05$ and the maximum coordinate
must be $\geq 0.95$. \\ 
{\bf Rxy} & & - & A factor that determines the width of the display
plots to their height when displaying the data. Standard is 1.5. \\
\end{tabular}

Note that the channel template file can contain many more channel
labels than the actual file. SPM searches, for each channel in the
data, through the labels in the channel template file. If the labels
match, the 2D-coordinate is used. 


\section{Reading of data}
\label{sec:load}
If you use the GUI only, there is no need to read this
section because the functions called by the GUI will read the data
automatically. However, if you plan to write scripts and access the
data and header information more directly, this section should contain
all the necessary information to do so. 

An SPM8 for M/EEG file can be read using the \textit{spm\_eeg\_load}
function. Without any arguments a file requester asks for the name of
the file. With a string argument $P$, \textit{spm\_eeg\_load(P)} will
attempt to read a file with the name $P$. The SPM-format stores the
binary data in a *.dat file. All header information are stored in a
*.mat file. This *.mat file contains a single struct named {\textit D}
which contains several fields. When using \textit{spm\_eeg\_load}, the
struct is transformed into an object, and the data are linked into this
object. The linking is done via memory mapping using \textit{file\_array}
objects. Note that the data should always be read using the routine 
{\textit spm\_eeg\_load}. The memory mapped data can be
addressed like a matrix (see below) which is convenient for accessing
the data in a random access way. However, a word of caution: If you
write new values to this matrix, the matrix is not only changed
in the object (in memory), but also physically on the hard
disk.  

You can also load the header struct using a
simple \textit{load} but this just returns the header struct, without the
data linked in, and any spm-functions won't know what to do with this
struct. In the following, we will describe the methods
that one can use on an M/EEG-object. Note that we will not describe
the internal format of the data here. This would be helpful only for
programmers who want to write spm-functions, because when simply
analyzing M/EEG there should never be a need to access the internal
format. However, for programmers, there is plenty of documentation
within the meeg-class function \textit{meeg}. 

\subsection{Syntax}
\textit{D = spm\_eeg\_load(P)}
\\

\subsubsection{Input}
The input string {\textit P} is optional and contains the file name of the
*.mat file.

\subsubsection{Output}
The output struct {\textit D} contains all header information about the
data. The data are memory mapped and can be accessed directly using
something like \textit{d = D(:,:,1)}. This command would put the first
trial over all channels and time points into the variable $d$. The
first dimension of $D$ is channels, the second peri-stimulus time, and
the third is trials. If the data are time-frequency transformed, there
would be four dimensions, where the frequency dimension is squeezed in
at the second position (i.e., channels/frequencies/time/trials). If
you wanted to change the values of the data, you would write something
like \textit{D(1,2,3) = 1;}, which would change the value of the first
channel, second time-point, and third trial to 1.


\section{Methods for the M/EEG object}

VLADIMIR: please describe methods: fiducials, selectdata, trialonset.
\\
\\
M/EEG methods are functions that operate on an M/EEG object, loaded
with \textit{spm\_eeg\_load}. These methods should not be used, if you just
want to analyze your data using the GUI or simple scripts. However, if
you write your own scripts or high-level functions that need to read
or manipulate the object, you need the methods. In the following, we
will provide details about most of the methods, which appear to be
stable by the beta-release date. The code for all methods
can be found in the \textit{@meeg} SPM directory. Most methods provide
some minimalist help text. In the following, we will assume that the
object variable is called \textit{D}, i.e. previously load by using \textit{D
  = spm\_eeg\_load;}.

\subsection{Constructor meeg}
The \textit{meeg} method is a constructor. Called without any arguments it
will produce a consistent, but empty object. In SPM, the constructor
is called when a struct has been loaded into memory by
\textit{spm\_eeg\_load}, and is transformed into an
MEEG-object. Importantly, the constructor also checks the consistency
of the object. 

\subsection{display}
This method will return, in the matlab window, some information about
the object, e.g., \textit{display(D)}. 

\subsection{Number methods}
These are methods which return the number of something; they count the
number of channels, etc. For example, to find out how many channels an
MEEG object contains, you would use \textit{D.nchannels}, where $D$ is the
object. Number functions are \textit{nchannels, nfrequencies, nsamples,
  ntrials}.



\subsection{Reading and manipulation of information}
There are a large number of methods that can be used to either read or
write some information. The method name is the same but it depends on
the arguments whether something is read or stored. For example, when
you use the method \textit{badchannels}, you can either type
\textit{D.badchannels}, which returns the indices of all bad channels. You
could also change information about specific bad channels, e.g.,
\textit{D.badchannels([43:55], 1)} will flag channels 43 to 55 as
bad. You 
  could also use \textit{D.badchannels([43:55], ones(1,13)}, i.e. you
  can 
  either use a scalar to change all channels listed, or supply a
  0/1-flag for each channel. There are other functions which use the
  same logic. In the following we will list these functions and
  describe briefly what they do but won't go into much detail. We
  trust that you can work it out using the badchannels-example. 

\subsubsection{chanlabels}
This method reads or writes the label of the channels (string). 

\subsubsection{chantype}
This method reads or writes the type of a channel (string). Currently,
the type can be one of the following: 'EEG', 'MEG', 'EMG', 'EOG', or
'Other'.

\subsubsection{conditions}
This method  reads or writes the name of the condition of an epoch
(string). 


\subsubsection{events}
VLADIMIR

\subsubsection{fname}
This method reads or writes the name of the mat-file, in which the
header information are stored. 

\subsubsection{fnamedat}
This method reads or writes the name of the dat-file, in which the
data are stored. 

\subsubsection{frequencies}
If the data has been transformed to time-frequency, this method reads
or writes the frequencies (Hz) of the data.

\subsubsection{fsample}
This method reads or writes the sampling rate of the data. In SPM, all
data must have the same sampling frequency.


\subsubsection{history}
This method can read or add to the history of a file. Usually, each
time a SPM function (e.g. like converting) does something to a data
set, the function name and arguments (possibly after collecting them
with the GUI) are stored in the history. Effectively, the history is a
documentation how exactly the data were processed. Of course, the
history function can also be used to replicate the processing, or
generate (modifiable) scripts for processing other data in the same
way.

\subsubsection{path}
This method reads or writes the path, under which the mat- and
dat-files are stored. 

\subsubsection{reject}
This method reads or writes the indices of rejected (bad) trials.

\subsubsection{timeonset}
This method reads and writes the stimulus onset time (number) of
epoched data. 

\subsubsection{transformtype}
This method reads and writes the type of the data transform
(string). For example, when the data are transformed to a
time-frequency represention, the transformtype is set to 'TF'. For
time data, this is 'time'.

\subsubsection{type}
This method reads and writes the type of the data (string). Currently,
this string can be 'continuous', 'single', 'evoked', or 'grandmean'.

\subsubsection{units}
This method reads and writes the units of the measurements
(string). The units are channel-specific, i.e., each channel can have
its own units.


\subsection{Reading of information}
Some methods can only read information but not change them. These are:

\subsubsection{condlist}
This method returns a list of condition labels. Multiple entries of
labels have been removed.

\subsubsection{coor2D}
This method returns the 2D-coordinates used for displaying or writing
sensor data to voxel-based images.

\subsubsection{dtype}
This method returns the type under which the data are stored in the
file\_array object. 

\subsubsection{eogchannels}
This method returns which of the channels are EOG channels.

\subsubsection{indsample}
This method returns the index of the sample which is closest to a
specific time point (ms).

\subsubsection{meegchannels}
This method returns the indices of all channels that are either of the
MEG or EEG type. 

\subsubsection{modality}
This method returns the modality of channels (MEG, EEG, etc.).

\subsubsection{pickconditions}
This method returns the indices of trials of a certain condition. The
condition must be supplied by its label (string).

\subsubsection{repl}
This method returns the number of replications measured for a
condition. This method is usually only used on single trial data.

\subsubsection{time}
This method returns the time (ms) of the samples.

\subsection{Manipulation of information}
There are two functions which only manipulate the objects. 

\subsubsection{ftraw}
This method converts an object to a fieldtrip struct. An additional
argument can be supplied to indicate whether the data is memory mapped
(1: default) or loaded into memory (0). Note that not all Fieldtrip functions
can properly handle memory mapped data. In order to avoid corrupting
the data, ftraw sets a read-only flag on it so in the worst case you might
encounter errors in Fieldtrip functions. Please report such errors on
SPM or Fieldtrip mailing list as we are interested in fixing them. ftraw(0) is 
safer in this respect but inefficient in its memory use. 


\subsubsection{save}
This method saves the object to the mat- and dat-files.


\section{SPM functions}
In this section we will describe the high-level SPM functions which
are used for preprocessing M/EEG data. These functions are fairly
standard and should allow a simple preprocessing of the data (e.g.,
epoching, filtering, averaging, etc.). Here, we will just describe
what each function roughly does and what the input arguments mean. More detailed information about the
syntax can be found in the help text of the code. For example, to get detailed help on epoching, type
\textit{spm\_eeg\_epochs}. The general syntax is the same for all
functions. If called from the command-line, and if no input arguments
are specified, the function will behave exactly as if you called the
function from the GUI by pressing a button or choosing it from the 'Other' menu. However, on the command line, or from a script,
you can supply input arguments, up to the point when all required input
arguments are specified, so that the function will run without any user
interaction. In this way, one can write a script that runs without
user interaction. See the folder \textit{man/example\_scripts} for an example. Input
arguments are provided in a struct $S$, whose 
fields contain the arguments. A typical call, e.g., from a script
would be: \textit{D = spm\_eeg\_epochs(S)}, where $S$ is the input
struct, and $D$ contains the return argument, the epoched MEEG object. Note that, with all SPM functions, the object is also always written to hard disk. The filenames of the mat- and dat-files are generated by prepending a single letter to the old file
name. In the example of epoching this would be an 'e'. The idea is that
by calling a sequence of functions on a file, the list of first
letters of the file name shows (roughly) which preprocessing steps were
called to produce this file. Note that another way of calling SPM
functions and specifying all input parameters is to use the new batch
interface.

\subsection{Epoching the data: spm\_eeg\_epochs}
Epoching cuts out little chunks of continuous data and saves them as
'single trials'. In M/EEG research, this is a standard data selection
procedure to remove long gaps between each single trial. For each
stimulus onset, the epoched trial starts at some user-specified
pre-stimulus time and 
end at some post-stimulus time, e.g.~from -100 to 400 milliseconds in
peri-stimulus time. The epoched data will also be baseline-corrected,
i.e.~the mean of the pre-stimulus time is subtracted from the whole
trial. The resulting event codes are the same as saved in the *.mat
file. The prepended output letter is 'e'.

The epoching function can deal with two different ways of specifying trials you want to epoch. The first is to specify explicitly where each trial is located in the measured time-series. The second is to identify trials using the stored labels in the file. For most users, the second way is the most convenient, but sometimes, when the stored triggers in the file do not relate to what you want to epoch, you should use the first. 
\\
In the first input way, you specify a $N \times 2$ so-called \textit{trl}-matrix, where each row contains the start and end of a trial. Optionally, you can enter, for each trial, an additional offset, which VLADIMIR. When you use the offset, you enter a $N \times 3$ \textit{trl}-matrix. The default for the offset is 0. In addition you have to specify conditionlabels (string), either one for each trial or one for all trials. You can also enter a 'padding' which will add time points before and after each trial to allow the user to later cut out this padding again. This is useful, e.g., for filtering epoched data, where one would otherwise, without padding experience 'edge effects'. 

For the second input way, one first defines the pre- and post-stimulus interval, and then let SPM identify the trials by their 'event type' and 'event value'. These are strings or numbers which the software run by the EEG or MEG vendor uses when generating the measurement file. If you don't specify parameters for the epoching function, a GUI will pop up, and present the found triggers with their status and value entries. These can sometimes look strange, but if you want to run a batch or script doing the epoching, you have to find out first what the type and value are. Fortunately, these tend to be the same over scanning sessions, so that you can batch multi-subject epoching using the types and values found in one subject. You also have to come up with a 'condition label' for each trial type, which can be anything you choose. This is the label that SPM will use to indicate the trial type of a trial at later processing stages.


For both methods of input you also have to set a (0/1)-flag (no/yes) whether you want to review the information for all trials after selecting them, when you want to make sure that all your trials are there. You should set the review-flag to 0, if you write a non-interactive script. You can also choose to save the trial definitions, for example, for re-use of another epoching of the same data.


\subsection{Filtering the data: spm\_eeg\_filter}
Continuous or epoched data can be filtered, over time, with a low-,
high-, stop- or bandpass-filter. SPM uses a Butterworth filter to do this. Note that
SPM uses the signal processing toolbox. This means that you have to
have this toolbox to filter data in SPM. Phase delays are minimised by
using matlab's {\textit filtfilt} function which filters the data
twice, forwards and backwards. The prepended output letter is 'f'.

When you use the function in GUI mode, will automatically use the Butterworth-filter. You can then choose among four different ways of how you want to filter your data: Lowpass, highpass, bandpass, and stopband. Depending on your choice, SPM will ask for the cutoff(s) in Hz.


\subsection{Artefact detection and rejection: spm\_eeg\_artefact}
Some trials not only contain neuronal signals of
interest, but also a large amount of signal from other sources like
eye movements or muscular activity. These signal components are
referred to as artefacts. In SPM, we use two simple automatic
artefact detection schemes. The first is thresholding the data and the
second is robust averaging. One can also choose to detect artefacts
manually by visualizing each trial using the display. Another option
is to use a more sophisticated artefact detection approach
(implemented by some other software) and supply that information to
SPM. In addition, one could also use a fieldtrip function to do the
artefact detection/removal, and convert the resulting fieldtrip struct
back to SPM.

Thresholding of the data is done in two passes. In the first pass, SPM
detects all instances, over trials and channels, where the
absolute value is higher than the threshold. If a channel has more
than a certain percentage of artefactual trials, it is defined as a
bad channel. In a second pass the thresholding is repeated, but
without taking into account any bad channels. A trial for which the
absolute data surpasses the treshold in some channel (excluding bad
channels) is then considered artefactual and flagged as a rejected
trial.

Note that the function only indicates which trials are artefactual or
clean and subsequent processing steps (e.g.~averaging) will take this 
information into account. However, no data is actually removed from
the *.dat file. The *.dat file is actually copied over without any
change. The prepended output letter is 'a'.

When you call the function, you are first asked whether you want to read your own artefact list. This gives you the opportunity to tell SPM which of your trials are artefactual, and which are clean. The two lists of trial numbers don't need to be complete, i.e., you can specify a few trials as artefactual or clean but SPM will still check for artefacts in the remaining trials. The next question is whether you want to use robust averaging for your data. This approach estimates weights, lying between 0 and 1, that indicate how much artefactual a trial is. Later on, when averaging to produce evoked responses, each trial is weighted by this number. For example, if the weight of a trial is close to zero, it doesn't have much influence in the average, and is effectively treated like an artefactual trial. If you choose robust averaging, you first have to choose an offset for the weighting function (JAMES), and then the width of a smoothing kernel of the residuals (JAMES).

SPM will next ask whether you want to threshold your channels. If you choose yes, you can then enter a list of channels, for which you want to check whether the absolute values of a trial surpass this threshold. Next, the threshold itself can be either a vector of thresholds, i.e., one for each channel, or a single threshold, which SPM uses for all channels.


\subsection{Downsampling: spm\_eeg\_downsample}
The data can be downsampled to any sampling rate. This is useful if
the data was acquired at a higher sampling rate than one needs for 
making inferences about low-frequency components. For example,
resampling from 1000 Hz to 200 Hz would cut down the resulting file
size to 20\% of the original file size. Note that SPM's downsampling
routine uses the matlab function {\textit resample}, which is part of
the signal processing toolbox. The prepended output letter is 'd'.

Here, you choose the new sampling rate (Hz) which must be smaller than the old sampling rate.

\subsection{Rereferencing: spm\_eeg\_montage}
Sometimes it is necessary to re-reference the data to a new
reference. In SPM this is done by specifying a weight matrix, which 
pre-multiplies the data. This is a general approach which 
allows one to re-reference to the average over channels, to single
channels, or any linear combination of channels, e.g. the average over
a pair of channels. The prepended output letter is 'M'.

When you call the function, you will first be asked whether you want to use a GUI or information read from afile to specify the montage. If you choose GUI, you will see, on the left hand side, the montage-matrix, where each row stands for a new channel. This means the labels in the left column describe the new labels. The old labels are on top, that means, each row contains weights how the old channels must be weighted to produce a new channels in the montage. On the right hand side, you see a graphical represention of the current matrix. The default is the identity matrix, i.e., the montage will not change anything. The concept is very general. For example, if you want to remove channels from the data, set the corresponding columns of these channels to all zero. VLADIMIR, can you describe two other examples, e.g. average reference, etc? When you changed the weights of the matrix, you can check the montage by pressing the button in the lower right below the figure.
\\
If you choose to specify the montage by 'file', you have to enter the filename of a mat-file, which contains three variables: 'labelnew' (labels of new channels), 'labelorg' (labels of original channels), and the montage-matrix 'tra' (tra as in transform).

Finally, you will be asked, whether you want to 'keep the other channels'. This means that VLADIMIR.


\subsection{Grand mean: spm\_eeg\_grandmean}
The grand mean is usually understood as the average of evoked
responses over subjects. The grand mean function in SPM is typically
used to do exactly this, but can also be used to average over multiple 
EEG files, e.g.~multiple sessions of a single subject. The averaged
file will be written to the same directory as the first selected
file. The prepended output letter is 'g'.

The function will ask you for the name of the output file. Note that in a script, by default, when you don't specify an output filname, SPM will generate a new file with the filename of the first selected file, prepended with a 'g'.

\subsection{Merge: spm\_eeg\_merge}
Merging several MEEG files can be useful for concatenating multiple
sessions of a single subject. Another use is to merge files and then
use the display tool on the concatenated file to be able to display
data coming from different files in the same graph. This is the
preferred way in SPM to display data together that is split up into
several files. The merged file will be written into the same directory
as the first selected file. The prepended output letter is 'c'.

The function will first check whether there are at least two files, and whether all data are consistent with each other, i.e., have the same number of channels, time points, and sampling rate. The function will also ask for a number of condition labels, one for each file and condition. This gives you the opportunity to rename labels. This might be useful when you merge files which contain the same conditionlabels, e.g. when you used several sessions for one subject but measured the same conditions in all files. In this case, it might be helpful to rename the conditions like 'condition A' to something like 'condition A, session 1', etc.



\subsection{Time-frequency decomposition: spm\_eeg\_tf}
\label{sec:tf}
The time-frequency decomposition uses a continuous
Morlet wavelet transform. The result is written to one or two result
files, one contains the instantaneous power and the other, optionally
written, the phase estimates. One can select the channels and
frequencies for which power and phase should be estimated. Optionally,
one can apply a baseline correction to the power estimates, i.e.~the
mean power of a pre-stimulus interval is subtracted from the power
estimates. For power, the prepended output letters are {\textit
  t1\_}, for phase {\textit t2\_}.

The function will first ask you, after selecting the M/EEG file, for a list of frequencies, which is a vector of numbers (Hz). For each of these frequencies, SPM will estimate the power and phase at each channel, time-point and trial. The next question is whether you want to remove the 'baseline' (i.e. the average power over some time-period in the pre-stimulus interval) from the time-frequency power estimates. If you choose yes, SPM will ask you for the time-period over which you want to form the baseline. Next, SPM will ask you for the so-called 'Morelet Wavelet factor'. The default is 7. The greater this number, the less resolution you have over time, but the higher the resolution is in frequency. Then you can select the channels for which you want to compute the time-frequency decomposition. SPM will then ask you whether you also want to estimate the phase, in addition to power. The final question is whether you want to 'collapse channels'. If yes, SPM will average the estimates (within power and phase) over all selected channels. 

\subsection{Averaging: spm\_eeg\_average}
Averaging of single trial data is the crucial step to obtain the
evoked response. By default, when averaging single trial data, single
trials are averaged within trial type. Power data of single trials
(see sec.~\ref{sec:tf}) can also be averaged by using the function
\textit{spm\_eeg\_average\_TF}. The preprended output letter is 'm'.


\subsection{Contrast of trials: spm\_eeg\_weight\_epochs}
As an extension to the averaging functionality, SPM can also be used
to compute linear combinations of single trials 
or evoked responses. For example, if you want to compute the
difference between two evoked responses, you supply a contrast vector
of $[-1; 1]$. Similarly, if you want to remove some trials from the
file, you can do this by using a contrast vector like $[1; 0]$ which
would write a new file with only the first evoked response. The
preprended output letter is 'm'. 

The function will first ask you to 'Enter contrasts'. This is a matrix where each contrast is given by a row of this matrix. For example, if you compute just one contrast, you have to enter a vector of the same length as the number of trial types in the file. Note that SPM will zero-pad this vector (or matrix) if you specify less contrast weights than you have trials. The next question is whether you want to 'Weight by num replications'. This is important when you use this function on single trials, where, typically, you have a different number of trials for each trial type. If you then choose to average over multiple trials, this option allows you to choose whether you want to form an average that is weighted by the number of measurements within each trial type. As compared to an average, where you implicitly first form the averages within trial type, and then average with equal weighting.

\section{Displaying data}
JEAN
